% This template can be used to generate a CV from a Markdown file with a
% YAML frontmatter. This template was created by Hendrik Erz <hendrik@zettlr.com>

% We're using the KOMA Article class
\documentclass[11pt,a4paper]{scrartcl}

% Some packages useful for the font handling
\usepackage{fontspec}
\defaultfontfeatures{Mapping=tex-text}
\usepackage{xunicode}
\usepackage{xltxtra}

% sectsty allows us to modify how the headings look.
\usepackage{sectsty}

% The xcolor package provides some color functionality
\usepackage{xcolor}

% Define default colors
\definecolor{AccentColor}{rgb}{0.5,0.0,0.4}
\definecolor{TextColor}{rgb}{0.2,0.2,0.2}

% Allow the user to redefine the colors
$if(accentColor)$
\definecolor{AccentColor}{rgb}{$accentColor$}
$endif$
$if(textColor)$
\definecolor{TextColor}{rgb}{$textColor$}
$endif$

% The sections being used will all be colored in accent
\partfont{\color{AccentColor}}
\sectionfont{\color{TextColor}}
\subsectionfont{\color{TextColor}}

% Document fonts
\setmainfont{Roboto}
\setsansfont{Roboto}

% Again, allow user-overridable fonts
$if(mainfont)$
\setmainfont{$mainfont$}
$endif$
$if(sansfont)$
\setsansfont{$sansfont$}
$endif$

% i18n / l10n
\usepackage{polyglossia}
\setdefaultlanguage{english}

% Allow the user to set the language (stolen from the Pandoc template)
$if(lang)$
\setmainlanguage[$for(polyglossia-lang.options)$$polyglossia-lang.options$$sep$,$endfor$]{$polyglossia-lang.name$}
$for(polyglossia-otherlangs)$
\setotherlanguage[$for(polyglossia-otherlangs.options)$$polyglossia-otherlangs.options$$sep$,$endfor$]{$polyglossia-otherlangs.name$}
$endfor$
$endif$

% For better page headers and footers
\usepackage{fancyhdr}
\pagestyle{fancy}
\fancyhf{} % No page header
\cfoot{\color{TextColor}\thepage} % Centered page number
% Prevent the rulers from showing up
\renewcommand{\headrulewidth}{0pt}
\renewcommand{\footrulewidth}{0pt}

% Obligatory URL stuff: Make sure links look like normal
% text and can be broken up to fit the page flow.
\usepackage[obeyspaces,spaces]{url}
\usepackage[breaklinks,hidelinks]{hyperref}
% This command makes URLs look like the rest of the text
\urlstyle{same}

% Packages
\usepackage{amsmath}
\usepackage{amsfonts}
\usepackage{amssymb}
\usepackage{graphicx}

% Margin options
\usepackage[
  left=3cm,
  right=3cm,
  top=2cm,
  bottom=3cm
]{geometry}


% The package enumitem is the central package for the CV sections.
% It is used to display a label in our accent color and also aligns
% the text properly.
% We're choosing a very narrow label width, so don't overdo it!
% Please also note, if you're customising this, that labelsep should
% be the opposite of itemindent, since otherwise all lines except the
% first one of an item's "text" will be wrongly indented.
\usepackage{enumitem}
\setlist[description]{labelwidth=2cm,itemindent=-1cm,labelsep=1cm,align=right,left=2cm,font=\color{AccentColor}}

% Remove all heading numberings (even from the custom body-text)
\makeatletter
\renewcommand{\@seccntformat}[1]{}
\makeatother

% Metadata for the PDF
\author{$if(name)$$name$$endif$}
\title{$if(name)$$name$$endif$ - CV}

\begin{document}
% Immediately switch to text color
\color{TextColor}

% Main heading is the name. Warn the user if they forgot the property.
\part*{$if(name)$$name$$else$Please provide a name!$endif$}

% We're using "minipages" to display the address versus contact info
% in a two-column layout. Note that the end of the first minipage
% MUST be on the same line as the begin of the second (which we do
% by using a percent-sign IMMEDIATELY after the end.)
\begin{minipage}[t]{0.48\textwidth}
$if(occupation)$
\textbf{$occupation$} \\
$endif$
% Print the address line(s)
$for(address)$$address$ $sep$ \\$endfor$
\end{minipage}%
\begin{minipage}[t]{0.48\textwidth}
\begin{flushright}
% Print contact info if there is some. Everything is treated as an URL.
$if(contact)$
  \textbf{Contact} \\
    $for(contact)$
    \url{$contact$}$sep$ \\
    $endfor$
$endif$
\end{flushright}
\end{minipage}

% Here comes the custom body text (the text from the Markdown document)
$body$

% Separate the CV header from the sections with a nice ruler.
{\color{AccentColor}\rule{\linewidth}{0.5mm}}

% Now come the sections
$for(sections)$
\section*{$sections.title$}

% NOTE: The "description" is being set up in the header with the setlist-command
\begin{description}

$for(sections.items)$
\item [$if(sections.items.label)$$sections.items.label$$endif$] $for(sections.items.text)$$sections.items.text$$sep$ \newline $endfor$ 
$endfor$

\end{description}
$endfor$

\end{document}